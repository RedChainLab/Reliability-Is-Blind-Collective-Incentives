\section{Related work}\label{sec:related_work}

The problem tackled by \sysname~is at the crossroads of multiple research domains, which we present and compare to our work in this section.

\subsection{Decentralized search mechanisms}

There are still relatively few works on decentralized search over decentralized resources, compared to those on more centralized approaches.
Moreover, their focus lies on the search itself and its properties (e.g., censorship- and bias-resistance), rather than on information front-running.
DHT-based solutions like HypeerCube~\cite{zichichi_towards_2021} and Ditto~\cite{keizerDittoDecentralisedSimilarity2023} do not guarantee result completeness but only a response from the fastest node. 
Censorship is particularly possible in HypeerCube~\cite{zichichi_towards_2021}, where only one or a few nodes are responsible for specific keywords, and can choose to not respond to requests, or to delay them.
TheGraph~\cite{Graph} is an indexing mechanism for Ethereum smart contracts, where resources are indexed based on their perceived value by curators. 
Again, completeness is not guaranteed.
Finally, DeSearch~\cite{liBringingDecentralizedSearch2021} provides completeness and integrity of results using TEEs, but is not resilient as-is to information front-running, as search providers have the power over the network interfaces that connect the TEE to the outside world.

\subsection{Transaction-front-running-resilient protocols}

\etienne{this section could be reduced significantly (the RW in general is a tad long).}
Some mechanisms deployed by protocols that protect against transaction front-running (TFR) in blockchain could be used against information front-running (IFR) in search systems, as both concepts are related: 
while the former aims to order a specific transaction before others, the latter delays some responses in favor of others.
Note that while solutions against TFR bear similarities with IFR in that they also play on the ordering and timing of events, IFR remains a different problem from TFR: TFR solutions cannot be used as-is to protect against IFR. 
We discuss below the extent to which their mechanisms can be repurposed to protect against IFR.

Torres et al.~\cite{torresFrontrunnerJonesRaiders2021} present the absence of transaction confidentiality and the blockchain miner's ability to choose transaction order as key causes for TFR.
To this end, transaction-ordering systems were proposed~\cite{bentovTesseractRealTimeCryptocurrency2019,kelkarThemisFastStrong2023,stathakopoulouAddingFairnessOrder2021}, ensuring that individual servers cannot influence the ordering of transactions.
Translating their protocols to our context, given such ordering mechanisms, existing acknowledgment messages sent back to the client can be repurposed as a response message to the client's request.
The response time after computation will then be that of the fastest server in the network. 
If the closest servers are malicious, then the client can still suffer from timing attacks, as is the case with \sysname.
Nonetheless, requiring consensus for a request-response scheme is computationally expensive (considering $n$ nodes, there is at least $O(n^2)$ in message complexity with TOB~\cite{cachinQuickOrderFairness2022}; or $O(n)$ optimistically with SNARK-THEMIS~\cite{kelkarThemisFastStrong2023}, but with added SNARK-proof computation time).

Fairy~\cite{stathakopoulouAddingFairnessOrder2021} also uses transaction sender anonymization through relays like TOR~\cite{TorProjectPrivacy}. 
Because malicious consumers can still decide to communicate directly with providers, they can still be recognized by malicious providers.
Meanwhile, honest responses are delayed through the high-latency layered network. 
As \Cref{ssec:res-malPL} shows, this creates an IFR disavantage for honest consumers by design, even without attacks.

The behavior of solutions presented above, which are originally tailored to protect against \emph{transaction front-running}, is captured by the ``multiprovider selection'' class defined in \Cref{sec:experimental_protocol}.
We have shown in \Cref{ssec:res-multi} that IFR has a strong impact in multiprovider-selection-based solutions.

\subsection{Oracle systems}

Blockchains operate as closed systems: they must rely on other entities to access data that is located outside their storage structure. 
``Oracle systems'' provide this dataflow between blockchains and the outside world.
Request-response search systems on top of blockchains~\cite{liBringingDecentralizedSearch2021,keizerDittoDecentralisedSimilarity2023,zichichi_towards_2021} can be seen as ``pull-outbound oracles''~\cite{muhlbergerFoundationalOraclePatterns2020}: they draw data from the blockchain and feed it to outside entities, in our case, the market consumers.
%Meanwhile, publish-subscribe search systems, left for future work, would be similar to push-outbound oracles.

If we consider the potential usage of our contribution for inbound oracles, i.e., those that feed data to blockchains, the block generation period (e.g., around 12s in Ethereum 2.0~\cite{etherscan.ioEthereumAverageBlock}) outweighs the subsecond gains from latency optimization.
Additionally, there is no obvious translation of inter-consumer competition in that context, so no IFR and therefore, no need to protect against it. 
This is why our proposed solution \sysname is rather suited for outbound oracles, that is, to serve users or systems that are not blockchains.
For more details, recent surveys explore existing works on blockchain oracle systems~\cite{caldarelliOverviewBlockchainOracle2022,al-breikiTrustworthyBlockchainOracles2020,heissOraclesTrustworthyData2019}. 
%The latter focuses on categorizing academic contributions to the field, while the first two propose an oracle system taxonomy and assess both industry and academic works. 

\subsection{Decentralized QoS-based provider selection}

Provider selection protocols are more in line with our work, as they indirectly tackle the IFR problem.
However, they usually operate in a non-adversarial setting, where providers are assumed to be honest and the only problem is to select the best one. 
In most cases, \sysname\space uses similar mechanisms to select providers, but adapts them to the adversarial setting with a trusted actor which is the TEE.
A close work with ours is Go-with-the-winner~\cite{liuGowiththewinnerPerformanceBased2016}, where providers are selected based on their end-to-end latency, in a non-adversarial setting, without insights from the providers themselves. 
%This protects the protocol from being influenced by malicious providers. 
We augment this work to a setting with malicious actors: we leverage TEEs notably to protect against content attacks, and use COoL selection against timing attacks.
sPOT~\cite{panigrahyAnalysisEvaluationProximitybased2022}, Spatial Power of Two, is based on a well-known mechanism called Power of Two Choices~\cite{mitzenmacherPowerTwoChoices2001}, where a client selects two providers at random and chooses the least loaded one.
In the case of sPOT, the two providers are the closest ones to the client. 
We show in \Cref{ssec:res-spot} that this protocol is subject to IFR attacks.
In \sysname, we use historic latencies instead of network distances to select providers.
%With \sysname, the TEE is able to provide a trustworthy indicator of its historic load, to use for later selection.
Other works, like DONAR~\cite{wendellDONARDecentralizedServer2010}, use an additional role in the protocol, a mapper or broker, to select providers. 
This adds communication costs that can be bypassed by malicious consumers, increasing by design their IFR advantage.
%\subsection{Tackling fair Quality-of-Service}
%What they provide as additional properties and at what cost (if any)?