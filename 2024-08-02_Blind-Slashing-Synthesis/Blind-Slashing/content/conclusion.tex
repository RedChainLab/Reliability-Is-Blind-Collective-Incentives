\section{Conclusion}\label{sec:conclusion}

In this paper, we highlight how providers of a decentralized marketplace's search service are able to give favorable treatment to part of the userbase, through \emph{information front-running} attacks, even if Trusted Execution Environments (TEEs) are used to guarantee the integrity and confidentiality of the search mechanism.

We show that quality-of-service-aware provider selection as performed by our proposed mechanism \sysname, on top of TEEs, is paramount to counter such malicious provider behaviour. 
Thanks to \sysname, we show that whether users collude or not with malicious providers, they do not get an edge over others in terms of fresher market state information, as long as there is enough honest computing power to serve all requests, in networks with low latency heterogeneity.
Future research directions include investigating the dynamic spawning or retirement of service providers over the network topology, tailored to the users' latency requirements to deal with the intrinsic heterogeneity of client latency distributions.
