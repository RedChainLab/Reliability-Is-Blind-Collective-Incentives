\section{\sysname\space | Detailed description}\label{sec:solution}

The key to selecting providers such that average round-trip latencies are minimized lies in the design of the \emph{Provider Selection Module}, which is detailed in this section.

\subsection{Available information}

The \emph{Provider Selection Module} has access to the sequences of observed round-trip latencies for each provider $j$, denoted $\{\Delta_{j}=\lbrack \Delta_{j}^{k}\rbrack | j\in\llbracket 1, N_{P}\rrbracket\}$, where $\Delta_{j}^{k}$ is the round-trip latency of the $k$-th request-response exchange between the \emph{Consumer} and \emph{Provider} $j$.
It also maintains the set of selection ratios, i.e, the probabilities to select each provider $j$, denoted $\{ r_{j} | j\in\llbracket 1, N_{P}\rrbracket\}$, such that $\sum_{j\in \llbracket 1,N_{P}\rrbracket}r_{j}=1$.

\subsection{Module design rationales}

As the incoming request load on a given provider increases, the queuing delay at its TEE increases, which in turn increases the round-trip latency for the requests it serves.
This means that a consumer may need to divide its target request throughput among several providers to minimize the average round-trip latency it experiences.
With this objective in mind, and based on currently available information, the \emph{Provider Selection Module} shall tune the selection ratios $\{ r_{j} | j\in\llbracket 1, N_{P}\rrbracket\}$.
In other words, given a provider $j$ and observed round-trip latencies $\Delta_{j}$ for this provider, as well as some scoring function $f$, if $f(\Delta_{j})<\text{avg}_{k\in \llbracket 1,N_{P}\rrbracket}(f(\Delta_{k}))$, i.e., if provider $j$ under-performs compared to the average of all providers, then $r_{j}$ should be decreased. 
Conversely, a higher-than-average score should increase the selection ratio $r_{j}$.
In the following, we will use a sliding-window average of size $s$ as the scoring function $f$, that is, we average the $s$ latest observed latencies. 

\subsection{Submodules description}

The provider selection operates in two distinct operations, the selection itself each time a request should be sent, at step 2 in \Cref{fig:system_arch},
and the update of the selection ratios, which may happen asynchronously to the search mechanism. 
In the implemented solution, the update of the selection ratios happens each time $s$ responses are received, with $s$ the sliding window size used in the ratio update algorithm.

\subsubsection{Provider selection}

At step 2 of the search mechanism (see \Cref{sec:overview} and \Cref{fig:system_arch}), the \emph{Provider Selection Module} selects a provider $j$ with a probability $r_{j}$, where the request will then be sent in step 3.

\subsubsection{Client-side Optimization of Latencies}

In a recurring fashion, e.g., each time $s$ responses are received, the \emph{Provider Selection Module} updates the selection ratios $\{ r_{j}\}$, based on the observed round-trip latencies $\{\Delta_{j}\}$, using \Cref{algo:update_ratios}.

\setlength{\textfloatsep}{0pt}
\begin{algorithm}
    \KwIn{$\{ r_{j}^{in} | j\in\llbracket 1, N_{P}\rrbracket\}$ the initial set of selection ratios for each provider $j$, such that $\sum_{j\in \llbracket 1,N_{P}\rrbracket}r_{j}=1$;
    $\{\Delta_{j}=\lbrack \Delta_{j}^{k}\rbrack | j\in\llbracket 1, N_{P}\rrbracket\}$ the set of sequences of observed round-trip latencies (ordered from newest to oldest latencies) for each provider $j$;
    }
    \KwData{
    $x$ the exploration coefficient;
    $K_{p}$ the PD-controller error-proportional coefficient; $K_{d}$ the PD-controller error-derivative coefficient;
    $s$ the sliding window size
    }
    \KwOut{$\{ r_{j}^{out} | j\in\llbracket 1, N_{P}\rrbracket\}$ the new set of selection ratios for each provider $j$, such that $\sum_{j\in \llbracket 1,N_{P}\rrbracket}r_{j}=1$;}
    
    \Begin{
        $\Delta_{j}^{avg} \leftarrow \frac{1}{s}\sum_{k=1}^{s}\Delta_{j}^{k} \quad \forall j\in\llbracket 1, N_{P}\rrbracket$\;
        $\Delta_{j}^{prev,avg} \leftarrow \frac{1}{s}\sum_{k=s+1}^{2s}\Delta_{j}^{k} \quad \forall j\in\llbracket 1, N_{P}\rrbracket$\;
        $\Delta^{avg} \leftarrow \frac{1}{N_{P}}\sum_{j=1}^{N_{P}}\Delta_{j}^{avg}$\;
    
        \For{$j\in\llbracket 1, N_{P}\rrbracket$}{
            $e_{j} \leftarrow \frac{\Delta^{avg} - \Delta_{j}^{avg}}{\Delta^{avg}}$\;
            $d_{j} \leftarrow \Delta_{j}^{avg} - \Delta_{j}^{prev,avg}$\;
            $r_{j}^{tmp} \leftarrow clamp(r_{j}^{in} + K_{p}e_{j} + K_{d}d_{j}$, 0, 1)\;
        }
    
        \For{$j\in\llbracket 1, N_{P}\rrbracket$}{
            $r_{j}^{norm} \leftarrow \frac{r_{j}^{tmp}}{\sum_{j\in \llbracket 1,N_{P}\rrbracket}r_{j}^{tmp}}$\;
            $r_{j}^{out} \leftarrow (1-x)r_{j}^{norm} + x\frac{1}{N_{P}}$\;
        }
        \Return{$\{ r_{j}^{out} | j\in\llbracket 1, N_{P}\rrbracket\}$}
    }
    \caption{UpdateSelectionRatios}
    \label{algo:update_ratios}
\end{algorithm}

In \Cref{algo:update_ratios}, we use the logic of a PD-controller (Proportional Derivative) to update the selection ratios, with the error-proportional coefficient $K_{p}$ and the error-derivative coefficient $K_{d}$. 
The setpoint, i.e., the target value, for this controller is the average score among all providers, i.e., the average round-trip latency among all providers, $\Delta^{avg}$, in the last $s$ request-response interactions with each provider.
The objective of the controller is to incrementally update selection ratios $r_{j}$ to bring provider-specific average latencies $\Delta_{j}^{avg}$ closer to the global average latency of all providers $\Delta^{avg}$.

In more detail, the algorithm proceeds as follows, and as illustrated by \Cref{algo:update_ratios}.
First, average round-trip times for the current and previous windows are computed for each provider $j$ (lines 2 and 3), as well as the global average round-trip time $\Delta^{avg}$ (line 4), the setpoint of the PD-controller.
Then, for each provider $j$, the error $e_{j}$ with respect to the setpoint and the derivative of the error $d_{j}$ are computed (lines 6 and 7).
$e_{j}$ is the relative difference between the provider $j$'s average latency and the global average latency. 
$d_{j}$ is the difference between the provider $j$'s average latency in the current window and the previous window.
Then, the selection ratio $r_{j}$ is updated using the PD-controller formula, and clamped so the resulting value is the nearest available in the interval $[0,1]$ (line 8).
The selection ratios are normalized (lines 10), i.e., $\sum r_{j} = 1$.
Finally, we remark that this client-side provider selection can be described as a Multi-Armed Bandit problem~\cite{slivkins2019introduction}: a consumer sends a request to a chosen provider $j$ (i.e., pulls lever $j$), and experiences a round-trip latency (i.e., the reward).
The consumer's objective is to maximize its reward: select providers such that it minimizes latencies. 
As is common in Multi-armed bandit settings~\cite{slivkins2019introduction}, the selection policy should allocate a small fraction of requests to explore the latency-performance of each provider, to avoid exploiting sub-optimal providers asymptotically.
This is achieved by mixing a uniform probability of selecting each provider $\frac{1}{N_{P}}$ with the current values of $r_{j}$, weighted by $x$ and $(1-x)$ respectively (line 11).

We draw attention to the fact that the presented algorithm requires calibration of the coefficients $K_{p}$ and $K_{d}$, as well as of the exploration coefficient $x$, so as to offer theoretical guarantees.
This can be online, during the system's operation, with parameter-tuning algorithms~\cite{fiducioso2019safe,xu2023config}. 
Algorithms from the Multi-Armed Bandit literature that tolerate evolving reward distributions, like Exp3~\cite{auer2002nonstochastic}, may also be used. 

As our focus was not on convergence, but on the asymptotic behavior of the system where selection ratios have converged, parameters were tuned using the Ziegler–Nichols method~\cite{ziegler1942optimum}.
We tune the system such that its convergence was near-optimal in micro-benchmark scenarios, i.e., in a set of stereotypical network scenarios (e.g., high- versus low-latency providers; high- versus low-throughput providers), and converged with random topologies drawn from the dataset which will be presented in the next section.
