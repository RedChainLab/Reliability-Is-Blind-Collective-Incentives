\section{Introduction}

%1 Establish a territory: bring out the importance of the subject and/or make general statements about the subject and/or present an overview on current research on the subject.
%2 Establish a niche: oppose an existing assumption or reveal a research gap or formulate a research question or problem or continue a tradition.
%3 Occupy the niche: sketch the intent of the own work and/or outline important characteristics of the own work; outline important results; and give a brief outlook on the structure of the paper.

%The introduction sets the story of the paper. It can be viewed like a funnel that takes on board all readers with different background, motivation and expectations and leads them to your contribution.
%An introduction has the following parts:
%• The motivation introduces the topic and claims the field. Very short.
%• The problem explains what you want to solve. Examples usually help the reader to create an early, intuitive understanding.
%• The contribution explain your contributions/solutions to the problems.
%Make them explicit, i.e. use a bullet point list. This is the most important part of the introduction.
%• The impact explains why your contribution is relevant

%It engages the reader by telling a story (but not your personal research story)
%• It starts fast and finishes strong
%• It answers the readers key question
%What is the main question addressed by the paper?
%Why is it important right now?
%What are the main contributions of the paper?
%What are the obtained results?
%Why should the reader care?
%• It clearly identifies what your work is by using active, explicit formulations (i.e. "our contribution is...")

Look at this: \cite{binmore_game_2007}