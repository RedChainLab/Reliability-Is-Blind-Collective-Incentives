%%
%% This is file `sample-sigplan.tex',
%% generated with the docstrip utility.
%%
%% The original source files were:
%%
%% samples.dtx  (with options: `all,proceedings,bibtex,sigplan')
%% 
%% IMPORTANT NOTICE:
%% 
%% For the copyright see the source file.
%% 
%% Any modified versions of this file must be renamed
%% with new filenames distinct from sample-sigplan.tex.
%% 
%% For distribution of the original source see the terms
%% for copying and modification in the file samples.dtx.
%% 
%% This generated file may be distributed as long as the
%% original source files, as listed above, are part of the
%% same distribution. (The sources need not necessarily be
%% in the same archive or directory.)
%%
%%
%% Commands for TeXCount
%TC:macro \cite [option:text,text]
%TC:macro \citep [option:text,text]
%TC:macro \citet [option:text,text]
%TC:envir table 0 1
%TC:envir table* 0 1
%TC:envir tabular [ignore] word
%TC:envir displaymath 0 word
%TC:envir math 0 word
%TC:envir comment 0 0
%%
%%
%% The first command in your LaTeX source must be the \documentclass
%% command.
%%
%% For submission and review of your manuscript please change the
%% command to \documentclass[manuscript, screen, review]{acmart}.
%%
%% When submitting camera ready or to TAPS, please change the command
%% to \documentclass[sigconf]{acmart} or whichever template is required
%% for your publication.
%%
%%

\documentclass[sigplan,10pt,anonymous,review,nonacm]{acmart}

\acmSubmissionID{142}
%\renewcommand\footnotetextcopyrightpermission[1]{}
%\settopmatter{printfolios=true,printacmref=false}

%%
%% \BibTeX command to typeset BibTeX logo in the docs
\AtBeginDocument{%
  \providecommand\BibTeX{{%
    Bib\TeX}}}

%% Rights management information.  This information is sent to you
%% when you complete the rights form.  These commands have SAMPLE
%% values in them; it is your responsibility as an author to replace
%% the commands and values with those provided to you when you
%% complete the rights form.
\setcopyright{acmlicensed}
\copyrightyear{2018}
\acmYear{2018}
\acmDOI{XXXXXXX.XXXXXXX}

%% These commands are for a PROCEEDINGS abstract or paper.
\acmConference[Middleware '24]{25th ACM/IFIP International Middleware Conference}{Dec. 02--06,
  2024}{Hong Kong, China}
%%
%%  Uncomment \acmBooktitle if the title of the proceedings is different
%%  from ``Proceedings of ...''!
%%
%%\acmBooktitle{Woodstock '18: ACM Symposium on Neural Gaze Detection,
%%  June 03--05, 2018, Woodstock, NY}
\acmISBN{978-1-4503-XXXX-X/18/06}


%%
%% Submission ID.
%% Use this when submitting an article to a sponsored event. You'll
%% receive a unique submission ID from the organizers
%% of the event, and this ID should be used as the parameter to this command.
%%\acmSubmissionID{123-A56-BU3}

%%
%% For managing citations, it is recommended to use bibliography
%% files in BibTeX format.
%%
%% You can then either use BibTeX with the ACM-Reference-Format style,
%% or BibLaTeX with the acmnumeric or acmauthoryear sytles, that include
%% support for advanced citation of software artefact from the
%% biblatex-software package, also separately available on CTAN.
%%
%% Look at the sample-*-biblatex.tex files for templates showcasing
%% the biblatex styles.
%%

%%
%% The majority of ACM publications use numbered citations and
%% references.  The command \citestyle{authoryear} switches to the
%% "author year" style.
%%
%% If you are preparing content for an event
%% sponsored by ACM SIGGRAPH, you must use the "author year" style of
%% citations and references.
%% Uncommenting
%% the next command will enable that style.
%%\citestyle{acmauthoryear}

\usepackage[ruled,vlined,linesnumbered]{algorithm2e}
\usepackage{subcaption}
\usepackage{hyperref}
\hypersetup{hidelinks=true, }
\usepackage[nameinlink]{cleveref}
\usepackage{fontawesome5}


 \newbool{showcomments}
 \setbool{showcomments}{false}
 \ifthenelse{\boolean{showcomments}}
 { \newcommand{\mynote}[3]{
    \fbox{\bfseries\sffamily\scriptsize#1}{\small$\blacktriangleright$\textsf{\emph{\color{#3}{#2}}}$\blacktriangleleft$}
    }}
 { \newcommand{\mynote}[3]{}}


\newcommand{\sonia}[2][]{\mynote{#1Sonia}{#2}{magenta}}
\newcommand{\matthieu}[2][]{\mynote{#1Matthieu}{#2}{orange}}
\newcommand{\etienne}[2][]{\mynote{#1Etienne}{#2}{blue}}

\newcommand{\awaitingreview}[1]{\faIcon{balance-scale}~}
\newcommand{\done}[0]{\faIcon{check-square}~}
\newcommand{\question}[0]{\faIcon{question-circle}~}
\newcommand{\plannedforlater}[0]{\faIcon[regular]{calendar-alt}~}
\newcommand{\wontdo}[1]{\faIcon{times}~}
\newcommand{\sysname}{COoL-TEE\xspace}
\newcommand{\figpatha}{"imports/img/wait500ByzRho100-WtA,+2250s-lat-dist-cumul-quantiles-SEND-RECV-0ms;0.00001x100,0;0.5;25.pdf"}
\newcommand{\figpathb}{"imports/img/acqshare-cons-behav-prov-behav-werr_500-SEND-RECV-0ms,50ms;0,0.00001x100;0.875,0.5,0.75,0.375,0,0.25,0.125,0.625;75;100.pdf"}
\newcommand{\figpathc}{"imports/img/wait500ByzRho100-noTEE,WtA,+2250s;WtA,+2250s-acqshare-cons-behav-prov-behav-werr_500.pdf"}
\newcommand{\figpathmodel}{imports/img/System Model.pdf}
\newcommand{\figpatharch}{imports/img/System Overview.pdf}
\newcommand{\figpathattack}{imports/img/Attack Model.pdf}
\newcommand{\figpathmotivation}{"imports/img/mal-acqshare-cons-behav-prov-behav-werr_500-SEND-RECV-100;0ms,50ms;0,0.00001x100;6I12;50.pdf"}
\newcommand{\figpathsentrq}{"imports/img/sentrq-cons-behav-prov-behav-share-werr_500-50ms,0ms;0,0.00001x100;0.5,0.375,0.75,0,0.625,0.875,0.125,0.25;75;100.pdf"}
\newcommand{\figpathnbsawt}{"imports/img/wait500Byz-WtA,+2250s;noTEE,WtA,+2250s-nbsa-cumul-wt-consprov-behav-SERVE-RECV-50ms;0,0.00001x10;0.5;25.pdf"}
\newcommand{\figpathcasemulti}{"imports/img/acqshare-cons-behav-prov-behav-plot-werr_500-SEND-RECV-100;0ms,50ms;0.00001x100,0;0,0.875,0.375,0.25,0.625,0.125,0.75,0.5;75,18.75,9.375,37.5.pdf"}
\newcommand{\figpathcaseba}{"imports/img/acqshare-cons-behav-prov-behav-plot-werr_500-SEND-RECV-50ms,0ms;0.00001x100;0.125,0.25,0.5,0.875,0.375,0,0.75,0.625;75;100.pdf"}
\newcommand{\figpathcasebb}{"imports/img/acqshare-cons-behav-prov-behav-plot-werr_500-SEND-RECV.pdf"}
\newcommand{\figpathcasebleg}{"imports/img/acqshare-cons-behav-prov-behav-plot-werr_500-SEND-RECV-50ms,0ms;0.00001x100;0.375,0.875,0.125,0,0.25,0.75,0.5,0.625;75;100-leg.pdf"}
\newcommand{\figpathcasespot}{"imports/img/acqshare-cons-behav-prov-behav-plot-werr_500-SEND-RECV-100;50ms,0ms;0.00001x100,0;0.625,0,0.75,0.25,0.375,0.875,0.125,0.5;75;true,false.pdf"}
\newcommand{\figpathlatthrough}{"imports/img/mean_latency_vs_total_throughput.pdf"}
\newcommand{\figpexodus}{"imports/img/pexodus.pdf"}


%%
%% end of the preamble, start of the body of the document source.
\begin{document}

%%
%% The "title" command has an optional parameter,
%% allowing the author to define a "short title" to be used in page headers.
\title{\sysname: Moderation of Multi-Party Interaction in Decentralized Systems with Limited Individual Information}

%%
%% The "author" command and its associated commands are used to define
%% the authors and their affiliations.
%% Of note is the shared affiliation of the first two authors, and the
%% "authornote" and "authornotemark" commands
%% used to denote shared contribution to the research.
\author{Henry Mont}
\email{henry.mont@insa-lyon.fr}
\affiliation{%
  \institution{LIRIS -- INSA Lyon}
  \city{Lyon}
  \country{France}
}

\author{Matthieu Bettinger}
\email{matthieu.bettinger@insa-lyon.fr}
\orcid{0000-0001-8129-7245}
\affiliation{%
  \institution{LIRIS -- INSA Lyon}
  \city{Lyon}
  \country{France}
}

\author{Sonia Ben Mokhtar}
\email{sonia.ben-mokhtar@cnrs.fr}
\affiliation{%
  \institution{LIRIS -- CNRS}
  \city{Lyon}
  \country{France}}


\author{Anthony Simonet-Boulogne}
\email{anthony.simonet-boulogne@iex.ec}
\affiliation{%
  \institution{iExec}
  \city{Lyon}
  \country{France}}

%%
%% By default, the full list of authors will be used in the page
%% headers. Often, this list is too long, and will overlap
%% other information printed in the page headers. This command allows
%% the author to define a more concise list
%% of authors' names for this purpose.
\renewcommand{\shortauthors}{Mont et al.}

%%
%% The abstract is a short summary of the work to be presented in the
%% article.
\begin{abstract}
  In decentralized cloud computing marketplaces such as the iExec marketplace, ensuring fair and efficient interactions between requesters, asset providers, and computing providers is crucial. 
  Traditional mechanisms often fail to address the complexity of task failures due to limited visibility into individual actor behavior. 
  This paper introduces \sysname, a novel approach designed to address these challenges by implementing a penalty system that broadly affects all involved parties when a task fails, rather than isolating penalties to the computing provider alone. 
  By leveraging a game-theoretic framework, we analyze the strategic behaviors of requesters, asset providers, and computing providers under the current system. 
  Then using ruin theory, we show that Blind Slashing incentivizes improved behavior across all actors by creating a disincentive for faults and misbehavior, leading to a more robust and fair marketplace. 
  This approach not only enhances the reliability of the system but also simplifies the management of task failures, providing a promising solution for decentralized environments where information is inherently limited.
\end{abstract}

% %%
% %% The code below is generated by the tool at http://dl.acm.org/ccs.cfm.
% %% Please copy and paste the code instead of the example below.
% %%
% \begin{CCSXML}
% <ccs2012>
%  <concept>
%   <concept_id>00000000.0000000.0000000</concept_id>
%   <concept_desc>Do Not Use This Code, Generate the Correct Terms for Your Paper</concept_desc>
%   <concept_significance>500</concept_significance>
%  </concept>
%  <concept>
%   <concept_id>00000000.00000000.00000000</concept_id>
%   <concept_desc>Do Not Use This Code, Generate the Correct Terms for Your Paper</concept_desc>
%   <concept_significance>300</concept_significance>
%  </concept>
%  <concept>
%   <concept_id>00000000.00000000.00000000</concept_id>
%   <concept_desc>Do Not Use This Code, Generate the Correct Terms for Your Paper</concept_desc>
%   <concept_significance>100</concept_significance>
%  </concept>
%  <concept>
%   <concept_id>00000000.00000000.00000000</concept_id>
%   <concept_desc>Do Not Use This Code, Generate the Correct Terms for Your Paper</concept_desc>
%   <concept_significance>100</concept_significance>
%  </concept>
% </ccs2012>
% \end{CCSXML}

% \ccsdesc[500]{Do Not Use This Code~Generate the Correct Terms for Your Paper}
% \ccsdesc[300]{Do Not Use This Code~Generate the Correct Terms for Your Paper}
% \ccsdesc{Do Not Use This Code~Generate the Correct Terms for Your Paper}
% \ccsdesc[100]{Do Not Use This Code~Generate the Correct Terms for Your Paper}

%%
%% Keywords. The author(s) should pick words that accurately describe
%% the work being presented. Separate the keywords with commas.
\keywords{decentralized cloud computing, decentralized marketplace, monitoring, limited information, collective punishment.}
%% A "teaser" image appears between the author and affiliation
%% information and the body of the document, and typically spans the
%% page.

% \received{x}
% \received[revised]{x}
% \received[accepted]{x}

%%
%% This command processes the author and affiliation and title
%% information and builds the first part of the formatted document.
\maketitle

\section{Introduction}

%1 Establish a territory: bring out the importance of the subject and/or make general statements about the subject and/or present an overview on current research on the subject.
%2 Establish a niche: oppose an existing assumption or reveal a research gap or formulate a research question or problem or continue a tradition.
%3 Occupy the niche: sketch the intent of the own work and/or outline important characteristics of the own work; outline important results; and give a brief outlook on the structure of the paper.

%The introduction sets the story of the paper. It can be viewed like a funnel that takes on board all readers with different background, motivation and expectations and leads them to your contribution.
%An introduction has the following parts:
%• The motivation introduces the topic and claims the field. Very short.
%• The problem explains what you want to solve. Examples usually help the reader to create an early, intuitive understanding.
%• The contribution explain your contributions/solutions to the problems.
%Make them explicit, i.e. use a bullet point list. This is the most important part of the introduction.
%• The impact explains why your contribution is relevant

%It engages the reader by telling a story (but not your personal research story)
%• It starts fast and finishes strong
%• It answers the readers key question
%What is the main question addressed by the paper?
%Why is it important right now?
%What are the main contributions of the paper?
%What are the obtained results?
%Why should the reader care?
%• It clearly identifies what your work is by using active, explicit formulations (i.e. "our contribution is...")

Key players in today's e-commerce landscape are logically centralized companies (e.g., Amazon, Alibaba, Ebay).
This centralization of power renders the marketplace vulnerable to attacks~\cite{skim}, failures~\cite{cuthbertsonFacebookUsersReport2021,swearingenWhenAmazonWeb2018}, and undesirable behaviors such as censorship and bias on goods proposed to consumers~\cite{CensorshipGoogle2024,gargSteemitCensoringUsers2019,glaserHowAppleAmazon2017}.

Novel decentralized marketplaces like OpenSea %\footnote{https://opensea.io}
 or Rarible %\footnote{https://rarible.com} 
 aim at solving these problems, notably by using blockchain and decentralized storage (e.g., IPFS~\cite{doan2022towards}) as building blocks. 
In such systems, anyone can offer assets for sale or look up assets to buy.
A match-making algorithm matches buy and sell orders emanating from each side of the market.

A key functionality for enabling buyers to find assets they want to acquire is a search mechanism.
Unfortunately, current decentralized marketplaces either lack an integrated search mechanism (e.g., OpenBazaar%\footnote{https://openbazaar.org}
), % a decentralized alternative to Ebay), 
or use a centralized search mechanism (e.g., OpenSea),
thereby reintroducing risks of censorship and bias. 

In recent years, solutions were proposed to enable decentralized search mechanisms in decentralized marketplaces~\cite{liBringingDecentralizedSearch2021,keizerDittoDecentralisedSimilarity2023,zichichi_towards_2021}.
Notably, DeSearch~\cite{liBringingDecentralizedSearch2021} proposes a multi-keyword search for decentralized services. %In particular, DeSearch enables to preserve trust during the entire search pipeline, from the resources hosted in the blockchain-based marketplace all the way to the response to a consumer query.
It leverages Trusted Execution Environments (Intel SGX~\cite{costan2016intel}) to protect buyers from censorship and bias attacks.
DeSearch relies on a subset of users contributing their computing assets to run the decentralized search protocol and act as service providers. 

\begin{figure}
  \centering

  \includegraphics[scale=0.8]{\figpathmotivation}

  \caption{\emph{Information front-running: without TEEs or without latency-aware provider selection, malicious consumers are able to be more often the first to see new assets} | Share of never-before-seen assets discovered first by malicious consumers ($\frac{50}{100}$ of total consumers), under different configurations. % (from left to right: without TEEs, with TEEs and random provider selection, with COoL-TEE, and in the fault-free case)
  $\frac{6}{12}$ search providers are malicious and the total system load is 50\%, see results in \Cref{sec:results} for a more exhaustive analysis.}% (Dis)advantage arrows pointing away from the 50% line
  \label{fig:motivation}\vspace{-0.5cm}
\end{figure}

In a context where assets are valuable, scarce, and often both, we must ensure that search service providers cannot favor some users and penalize others.
We are particularly interested in the case where service providers manipulate access to knowledge about new assets for some users, thereby giving a head-start to others.
We call this phenomenon an \textit{information front-running} (IFR) attack.
We show in this paper that recent work on decentralized search mechanisms for decentralized marketplaces is vulnerable to IFR attacks:
malicious service providers have the power to delay responses to honest consumers, in favor of malicious consumers, which we highlight as a novel attack in the context of TEEs.
Malicious consumers can accentuate the providers' attack by loading honest providers, increasing those providers' end-to-end latency, so as to steer honest consumers towards malicious consumers who will then attack them.

We illustrate the impact of IFR in \Cref{fig:motivation}: 
in an open system (1st bar) malicious providers and malicious clients can gain significant advantage due to their capacity to perform IFR and content attacks. 
DeSearch (2nd bar) prevents the latter by using TEEs, but is still vulnerable to delay attacks. 
\sysname (3rd bar) prevents delay attacks and is the closest to the behavior of a fault-free system (4th bar).

We propose \sysname, Client-side Optimization of Latencies coupled with TEEs, a mechanism to protect decentralized search systems from highlighted information front-running attacks.
Building upon the DeSearch~\cite{liBringingDecentralizedSearch2021} protocol, \sysname adds a \emph{Provider Selection Module} at the client-side, that selects providers based on past experiences with them, with respect to observed request-response round-trip latencies. 
Consumers are then able to steer away from providers with higher latencies, be it due to network distance or malicious behavior, and send their requests to providers that minimize experienced latencies. 
The proposed solution is client-side for simplicity and deployability: it does not require server-side modifications, coordination, or consensus.

%\sonia[\plannedforlater]{describe how \sysname is evaluated and its main results}
We evaluate \sysname in both a deployed cluster and a simulated environment (source code available~\cite{coolTEEcode}), in network setups featuring homogeneous or heterogeneous latencies between consumers and providers.
We compare the impact of \sysname and related work in an extensive study, in different system configurations and under different attacker strengths.
In particular, we evaluate the competitive advantage (or lack thereof) for a subset of consumers, and overall latency Quality-of-Service.
We show that in networks where latency heterogeneity between users is low, with \sysname, malicious users do not get an edge over others in terms of fresher market state information, as long as there is enough honest computing power to serve all requests. 
We also highlight how high heterogeneity in network latencies between users can give an intrinsic advantage to users who are already close to providers, even without attacks.

The remainder of the paper is structured as follows. 
We present in \Cref{sec:problem_statement} a model of the ecosystem and actors involved in decentralized marketplace search, upon which we then define our threat model. 
\Cref{sec:overview,sec:solution} provide an overview followed by a detailed description of \sysname. 
Next, in \Cref{sec:experimental_protocol,sec:results}, we respectively present our experimental protocol and our results.
In \Cref{sec:related_work}, we present related work, before concluding in \Cref{sec:conclusion}.

\section{Model \& problem statement}\label{sec:problem_statement}

We detail the model of decentralized marketplace search that we consider and formalize our system and attacker models.

\etienne[\done]{I would merge this section 2 with the following section 3, since the actual formalization of the attack is in section 3 and not in section 2. The content of section 2 can be summarized in the header as a brief overview (top-down) of the content of the new, merged section.}

% In this section, we first provide the model of decentralized marketplace search that we consider in this paper.
% We then further formalize our system and attacker models.

\subsection{Roles \& interactions}

\begin{figure}
    \centering
    \includegraphics[scale=0.8]{\figpathmodel}
    \caption{\emph{Marketplace search model} | Consumers send requests to Providers of the search mechanism about the market's state; Providers then respond with a list of assets they previously learned from the Market over time.}\vspace{-0.5cm}
    \label{fig:system_model}
\end{figure}

We assume the system is composed of the following three entities, also illustrated in Figure~\ref{fig:system_model}: 
(i)~the \emph{market} (represented in the right part of the figure) is the blockchain- and/or decentralized storage-based system in which assets are registered for future transactions, for example, to be sold or rented. 
\emph{contributors} of assets create the offer side of the market. 
In the scope of this paper, contributors and their actions are abstracted as a stream of \emph{market events} towards \emph{providers}; 
(ii)~\emph{consumers} (depicted in the left part of the figure) are users of the search mechanism;
and (iii)~\emph{providers} (depicted in the middle of the figure) provide their computing power to run the search mechanism: 
they act as intermediaries between the market and consumers.
To do so, providers update a local index of the market state according to the market events they receive, which they use to respond to consumers' search requests.

Some providers may collude with some consumers and offer these better treatments, typically in the exchange of bribes.
These actors are considered malicious.
The objective of malicious consumers is to maximize the probability that they are the first to see new assets before the rest of the consumer base.
We call Never-Before-Seen assets (NBSa) assets that are not yet discovered by any consumer.
NBSa become discovered NBSa (dNBSa) by some consumer once a response that contains those NBSa arrives at that consumer.
We refer to attacks that impact a consumer's share of dNBSa compared to others' as \emph{information front-running} (IFR) attacks.

\subsection{System formalization \& constraints}

As illustrated by \Cref{fig:system_model}, we consider $N_{C}$ consumers and $N_{P}$ providers placed on a network topology.
The communication delay between a consumer $i$ and a provider $j$ is $\delta_{i,j}$ (also referred to as $\delta_{net}$ as a generic notation).
Consumers may be co-located with providers, in which case we assume that communication is negligible compared to queuing and computation delays, i.e., $\delta_{i,j}=0s$.

We assume that two protocols run in parallel: the market and the search mechanism. 
Messages sent as part of those protocols will be represented in the format $\mathbf{<type:data>}$ (or the short-hand $\mathbf{<type>}$). 
On the market side, we assume that new assets emanate from the market towards providers, at an average rate $\lambda_{a}$.
For each new asset, $\mathbf{<NEW>}$ messages are broadcast to all providers. We assume (for the sake of simplicity) that all providers receive $\mathbf{<NEW>}$ messages at the same time.\etienne[\done]{revise to avoid passive tense.}
The representation of assets in $\mathbf{<NEW>}$ messages contains all the metadata necessary for subsequent indexing (e.g., its defining keywords). 

Concurrently, as part of the search mechanism, consumers send requests $\mathbf{<RQ>}$ to providers.
Requests contain a list of keywords of interest to the consumer.
Each consumer sends requests at a rate $\lambda_{r}$, each request to a single provider.
The provider handles those $\mathbf{<RQ>}$ messages following a First-In-First-Out policy. 
Specifically, messages wait in the queue for a delay $\delta_{q}$, depending on the queue size at arrival. 
When a request $\mathbf{<RQ>}$ is handled, the provider searches their local index of assets according to the request's keywords.
A fixed response time $D$ is required to handle the request. 
Then, they respond with a message $\mathbf{<RSP>}$, containing a list of assets that matched the filters. 
Note that only assets whose $\mathbf{<NEW>}$ message arrived before the consumer's request can be present in $\mathbf{<RSP>}$ messages.

\subsection{Building blocks}

We now describe the protocols and building blocks forming the foundation of this work. 

\subsubsection{Trusted Execution Environments}

Our work and the DeSearch protocol described below leverage Trusted Execution Environments (TEEs), namely Intel SGX~\cite{costan2016intel}, to guarantee the integrity and confidentiality of the search protocol's computations and communication.

TEEs are hardware-based security components integrated in CPUs.
They enable running code in a protected environment, called an enclave.
An enclave is isolated from the rest of the system including the operating system and other applications. 
This enclave can be remotely attested, e.g., by a consumer, to prove that the code running inside it is the expected one and is indeed running in a genuine TEE environment. 
Following this attestation, a secure channel can be established between the consumer and the enclave, to ensure confidentiality and integrity of their communications.
In our context, this means that the host, which may be malicious and is, therefore, not trusted, is not able to modify the content of the responses sent by the enclave.
However, the host is still able to delay the responses, once they are outside the TEE, a property that we exploit in this paper. 

\subsubsection{DeSearch~\cite{liBringingDecentralizedSearch2021}}

DeSearch is a decentralized search protocol for decentralized marketplaces, that leverages TEEs to protect consumers from censorship and bias attacks.
It is an epoch-based protocol that follows a 3-step pipeline: new assets registered on some verifiable storage, like blockchains or IPFS~\cite{doan2022towards}, are first crawled; these crawled assets are then indexed in the following epoch; and finally, ``Queriers'', our providers, serve search requests from clients, our consumers, in the third epoch. Our work focuses on the last step of this pipeline: the index created at the end of DeSearch's second step includes the new assets from the market.
This index is then fetched and served by providers.
That is, in this paper, we abstract DeSearch's phases upstream of the search mechanism as the market (right side of \Cref{fig:system_model}).

DeSearch uses TEEs to guarantee that results are complete and tamper-proof, such that providers cannot censor or bias results.
Additionally, it is necessary to hide accesses to the index of assets in memory from the possibly malicious host, as observing memory access patterns may allow it to infer the content of requests and perform targeted censorship.
In DeSearch and our work this threat is prevented by the use of Oblivious RAM, specifically Circuit-ORAM~\cite{wangCircuitORAMTightness2015} for accessing the index of assets.
Note that while ORAM randomizes memory access patterns, it also prevents the parallelization of serving requests across multiple threads accessing the same shared index~\cite{wangCircuitORAMTightness2015}.
As a result, both for DeSearch and \sysname it is assumed that requests are served sequentially.

\subsection{Threat model}

We assume that a fraction $c_{M}$ of consumers and $p_{M}$ of providers are malicious in the system. 
In the following, for the purpose of intuitive exposition, we consider that half the consumer-base is malicious ($c_{M}=50\%$) unless otherwise specified.
Specific attack behaviors are discussed in later subsections.
All other actors are assumed to be honest and they therefore follow the protocol correctly.

We assume that Providers use TEEs to protect the integrity of the search protocol as well as the confidentiality of the request/response messages. 
%In that case, the content of messages is considered tamper-proof, and the confidentiality of request and response messages in transit is ensured. 
%We consider for comparison purposes a baseline that does not use TEEs and therefore does not guarantee integrity and confidentiality.

We assume that malicious actors have similar available computing power compared to honest actors with the same role (e.g., between consumers).
In total, this computing power is proportional to $c_{M}$ and $p_{M}$. 
We assume that honest and malicious consumers have a respective request budget proportional to their fraction of the total population, e.g., proportional to $c_{M}$ for malicious consumers. 

\subsubsection{Adversarial objective}

We assume that malicious providers gain more when malicious consumers can acquire assets instead of honest ones (e.g., through bribes). 
The common objective for malicious actors is to maximize their share of discovered Never-Before-Seen assets (NBS-assets) compared to honest consumers.
We define that an asset $A$ was discovered first by a given consumer if a response $\mathbf{<RSP>}$ containing that asset was received by that consumer before any other consumer. 
The strength of the \emph{information front-running} experienced by the malicious faction in a given time frame is measured by the fraction of NBS-assets discovered first by malicious consumers, compared to the total number of new assets. 
The higher above $c_{M}$ this fraction is, the bigger the advantage malicious customers have over honest ones in future market transactions.

Given that both honest and malicious consumers compete over the same set of assets, preventing honest consumers from discovering NBS-assets first is a viable strategy for the malicious faction. 
The means available to the malicious faction are discussed in the following subsections and illustrated by \Cref{fig:attack_model}.

\subsection{Attacks formalization}

Based on the system constraints and the malicious faction's objective, we now define attacks led by malicious actors.

\begin{figure}
    \centering
    \begin{subfigure}{\linewidth}
        \centering
    \includegraphics[scale=0.8,page=1]{\figpathattack}
        \subcaption{Fault-free | 
        Consumer and Provider exchange a request for a response containing a list of assets.}
        \label{fig:fault_free}
    \end{subfigure}
    \begin{subfigure}{\linewidth}
        \centering
        \includegraphics[scale=0.8,page=2]{\figpathattack}
        \subcaption{Content attack | 
        Without TEEs, a malicious provider can send back fake or old assets to hinder asset discovery by honest consumers.}
        \label{fig:content_attack}
    \end{subfigure}
    \begin{subfigure}{\linewidth}
        \centering
        \includegraphics[scale=0.8,page=3]{\figpathattack}
        \subcaption{Timing attack |
        Content attacks (\Cref{sub@fig:content_attack}) are not feasible with TEEs, but providers can still delay responses so honest consumers receive staler market information than malicious ones.}
        \label{fig:timing_attack}
    \end{subfigure}
    \caption{Provider-side attack models on decentralized marketplace search}\vspace{-0.5cm}
    \label{fig:attack_model}
\end{figure}

\subsubsection{Content attacks} 

To compare the IFR attack impact in systems with and without TEEs, a strawman approach without TEEs is considered where malicious actors can have the most impact on the system. 
Indeed, without TEEs, malicious providers are able to read requests plainly and to modify (e.g., to fake) the content of the corresponding responses (\Cref{sub@fig:content_attack}), instead of sending only real assets as in the fault-free case (\Cref{sub@fig:fault_free}). 
Consequently, in the worst case, if an honest request is served by a malicious provider, it cannot be used to discover new assets.
As a result, for the same number of sent requests, malicious consumers have more opportunities to discover new assets than honest consumers.
We refer to this worst-case as \emph{content attacks}.
%Note that this attack becomes impossible in search systems with TEEs and verifiable result completeness, like DeSearch~\cite{liBringingDecentralizedSearch2021}.
%We present content attacks to be able to compare the IFR attack impact in systems with and without TEEs.

\etienne[\done]{Is it important to present this attack after saying that we build upon DeSearch? I could be simpler to mention in the DeSearch subsection in much shorter form saying that DeSearch prevents content attacks with a brief def. and that they are not the concern in this work.}

\subsubsection{Timing attacks}

%Using TEEs upholds the integrity and confidentiality of the search protocol, meaning honest and malicious consumers both receive correct responses and, considering an equal number of sent requests, they have the same number of opportunities to discover new assets. 
While TEEs uphold the integrity and confidentiality of the search protocol, TEEs do not prevent malicious providers from delaying responses to honest consumers.
This is a novel attack that TEEs cannot solve by themselves, where the surrounding host may arbitrarily delay messages coming in or out of the TEE, based on the message sender or recipient.\etienne[\done]{from beginning to end of paragraph: can be simplified/shortened significantly.}
Indeed, malicious providers can delay sending responses back to honest consumers by some delay $\delta_{att}$ (\Cref{sub@fig:timing_attack}), while they respond to malicious consumers immediately after computation, like the fault-free behavior (\Cref{sub@fig:fault_free}). 
Due to the TEE's computation integrity and its secure channel with the consumer, the response is correct and tamper-proof. 
Therefore, $\delta_{att}$ should be applied after the response is computed to lower its freshness.\etienne[\done]{Clarify why as this may not be clear immediately for all readers (computation integrity).}
%Staler responses to honest consumers contributes to the information front-running advantage experienced by malicious consumers, that receive fresh responses from any provider.
We refer to this behavior as a \emph{timing attack}.
We consider that malicious providers do not drop $\mathbf{<NEW>}$ messages.
With DeSearch~\cite{liBringingDecentralizedSearch2021}, search results are verifiably complete each epoch.

\subsubsection{Cuckoo attacks}

In this attack, malicious consumers target honest providers by overwhelming them with requests in order to slow them down.
Consequently, honest consumers who choose providers with lower latencies will leave honest providers in favor of malicious providers, who give them staler information through timing attacks.
We call this consumer-side attack a \emph{cuckoo-timing} attack.
We also evaluate cuckoo attacks on top on content attacks, i.e., \emph{cuckoo-content attacks}, for comparison purposes.

\section{\sysname\space | Overview}\label{sec:overview}

In \Cref{subsec:search}, we focus on the timing aspects we consider for the two protocols that operate in parallel in the considered decentralized marketplace search mechanism, as illustrated in \Cref{fig:system_arch}: 
the market indexing mechanism (right), as well as our proposed provider-selection and search mechanisms (left).\etienne[\done]{The indexing and search were already discussed in the previous sections, so it may be surprising to read that these are yet more mechanisms. Maybe say you detail them?}
We then describe the key principles of the main module involved in \sysname's search mechanism in \Cref{subsec:cool}: the \emph{Provider Selection Module} (PSM).

%\subsection{Market indexing mechanism}\label{subsec:market}
%\etienne{consider pruning redundancies with earlier similar descriptions (section 3.1)}
%The right half of \Cref{fig:system_arch} illustrates the process through which providers update their view on the market's state, in two alternating steps.
%(Step a) All providers receive a message $\mathbf{<NEW>}$ that a new asset appeared on the market. 
%Providers obtain market events $\mathbf{<NEW>}$ in periodic batches, that are verifiably complete and tamper-proof.
%(b) In reaction, based on these $\mathbf{<NEW>}$ messages, each provider $j$ updates its local market index.

\subsection{Protocol delays and timing}\label{subsec:search}

\etienne[\done]{This is quite redundant with content in section 3. It is also more background material than a contribution of the paper. It could be turned into a timing analysis without repeating the protocol already described previously.}
We focus here on the timing of our protocol steps.
\Cref{fig:system_arch} shows the behaviour of a consumer $i$ that uses the search mechanism and providers 1 to $N_{P}$ who provide this service: 
(step 1) at $t_{gen}$, the consumer $i$, on the left of \Cref{fig:system_arch}, first generates a request message $\mathbf{<RQ>}$ .
(2) The request is directed by the PSM to some provider $j$ at $t_{send}$, after a computation delay $\delta_{PSM}$. 
(3) The request is sent through the network to the chosen provider $j$, with a delay $\delta_{i,j}$. 
The untrusted host then receives the request message and forwards it to the TEE, who enqueues it.
(4) The request waits for $\delta_{queue}$ in the TEE's queue until service of this request begins at $t_{serv}$.
The time to serve the request after $t_{serv}$ is $\delta_{serv}=D$.
(5) After service completion, the TEE sends the response message $\mathbf{<RSP>}$ out to the untrusted host, who, if it is malicious, may delay the response by $\delta_{att}$. 
The response is then sent through the network to the consumer $i$, with a return-trip delay $\delta_{i,j}$.
(6) At $t_{recv}$, the consumer $i$ finally receives the response message $\mathbf{<RSP>}$ and the PSM logs the experienced round-trip latency, for future interactions. 
Let $k$ be the number of request-response interactions between consumer $i$ and provider $j$. 
Consumer $i$ logs $\Delta^{k}_{i,j}=t_{recv}-t_{send}=\delta_{queue}+D+2\delta_{i,j}+\delta_{att}$.

Concurrently, on the right of \Cref{fig:system_arch}, as a new asset $A$ appears on the market, $\mathbf{<NEW>}$ messages are broadcast to all providers (step a). 
Then, in step b, we consider that $A$ is added to the index at the same $t_{index}$ for all providers.
For $A$ to appear in a given $\mathbf{<RSP>}$ message, whose request would match $A$, $A$ must be indexed before serving the request: we require $t_{index}<t_{serv}$.
Furthermore, due to the TEE's computation integrity, an indexed asset $A$ will appear in responses if it matches the request.
Therefore, from a consumer's point of view, a response's freshness with respect to the market's state is the latency once service starts: $\delta_{fresh}=D+\delta_{i,j}+\delta_{att}$.
Honest consumers can maximize the freshness (i.e., minimize $\delta_{fresh}$) by choosing close providers in the network (low $\delta_{i,j}$), with high computation power (low $D$), and who are honest ($\delta_{att}=0$).
%\sysname does so for each consumer $i$ individually, by selecting providers so as to minimize the average of $\Delta^{k}_{i,j}$ round-trip latencies, which contain $\delta_{i,j}$ and $\delta_{att}$, experienced across all providers $j\in\llbracket 1,N_{P}\rrbracket$.

\begin{figure}
    \centering
    \includegraphics[trim={2mm 0 0 0},scale=0.95]{\figpatharch}
    \caption{High-level system architecture}\vspace{-0.5cm}
    \label{fig:system_arch}
%    \Description[]{}
\end{figure}

\subsection{COoL | Client-side Optimization of Latencies}\label{subsec:cool}

The key mechanism in \sysname is the Provider Selection Module (PSM). 
Before providing a detailed description, we first introduce the PSM's key principles.
A consumer wants the freshest information about the market, that is, responses that contain the newest relevant assets.
%The content of responses is immutable once they are generated thanks to TEEs, so freshness is achieved by minimizing the time between response generation at the provider and response reception at the consumer (i.e., the latency of step 5 in \Cref{fig:system_arch}). \etienne{It might not be clear that once the market event has been received by the provider, it has been processed and the information is in the index and the TEE ensures it cannot be ignored.}
As a secondary goal, consumers also want to minimize their request-response latencies, for user experience purposes.

As a consequence, the goal of a consumer's PSM is to prefer sending requests to providers with whom the consumer experiences low round-trip latencies, and in particular to avoid malicious providers leading timing attacks ($\delta_{att}>0$). 
Each consumer $i$'s PSM does so individually, by performing a provider selection based on \emph{Client-side Optimization of Latencies} (COoL): 
it aggregates each past request $k$'s round-trip latency $\Delta^{k}_{i,j}$ with each provider $j$, then tunes the probability to select each provider for future requests, so as to decrease the average round-trip request latency across all providers over time.
Considering a target throughput $\lambda_{r}$ of requests to send, consumer $i$ divides this throughput such that each provider $j$ receives a share $\lambda_{i,j}=r_{i,j}\lambda_{r}$ of the requests (with $\sum_{j\in \llbracket 1,N_{P}\rrbracket}r_{i,j}=1$).
At step 2 of the search mechanism (see \Cref{fig:system_arch}), the goal of the \emph{Provider Selection Module} is to tune all $r_{i,j}$ such that the average round-trip latency is minimized.
Doing so, \sysname's PSM indeed chooses providers with low round-trip latencies. 
Moreover, if an honest and a malicious provider $j$ and $j'$ have the same service time $D$ and network latencies $\delta_{i,j}=\delta_{i,j'}$ to consumer $i$, then consumer $i$ will preferentially choose the honest provider, whose $\delta_{att}=0$.
%\etienne{This overview details the solution, but not why it addresses the problem. I think this is what reviewers will be looking for.}

\section{\sysname\space | Detailed description}\label{sec:solution}

\section{Experimental protocol}\label{sec:experimental_protocol}

We present in this section the performance evaluation of \sysname, performed in terms of IFR advantages, as well as overall Quality-of-Service regarding roundtrip search latencies.
In particular, this section allows answering the following questions: 
\begin{itemize}
    \item[\emph{RQ1}:] What is the performance of \sysname\space in the fault free case?
    \item[\emph{RQ2}:] What is the performance of \sysname\space in an adversarial setting where providers are equally distant in terms of network latency from consumers (e.g., all providers are deployed in the same datacenter)?
    \item[\emph{RQ3}:] What is the performance of \sysname\space in an adversarial setting where providers are deployed over the internet, i.e., with heterogeneous network distance from consumers?
\end{itemize}
We answer these questions in \Cref{ssec:res-FFPL,ssec:res-sameDC,ssec:res-malPL}, respectively.
\subsection{Metrics}

Two types of metrics are used to assess the performance of \sysname: discovered Never-Before-Seen assets (dNBSa) and latencies.
The main metric is the number of discovered Never-Before-Seen assets (dNBS-assets) by malicious consumers, as defined in \Cref{sec:problem_statement}.

If the proportion of dNBS-assets discovered by any subpopulation of consumers is similar to the proportion of requests sent by that subpopulation, then the subpopulation has no IFR advantage over other subpopulations.
If that proportion is higher (resp. lower), then the subpopulation has a higher (resp. lower) IFR advantage (resp. disadvantage) over other subpopulations.

The secondary metric is the latencies experienced during the lifetime of consumer requests and their responses from providers. Indeed, latencies at specific steps of the protocol have an impact on the first metric.
Notably, the freshness of a response, i.e., the time elapsed between the moment the service of a request starts and the moment the response is received by the enquiring consumer, is the target of \emph{timing attacks}.
Therefore, we also observe the latencies of search protocol phases and those introduced by attacks, to understand their impact on NBS-asset discovery.

Roundtrip latencies will also be considered from the point-of-view of the Quality-of-Service (QoS) experienced by consumers, with lower latencies using a given mechanism indicating better latency-QoS.

\subsection{Deployment and simulation setups}

We implement the \sysname protocol in C++, using DeSearch~\cite{liBringingDecentralizedSearch2021} as the back-end search mechanism. 
Search providers run on an SGX-enabled 4-core Azure \emph{DC4s\_v2} VM, while consumers run on a non-SGX 16-core Azure \emph{B16als\_v2} VM.
Both machines are in the same datacenter, with a 1.1ms (0.5ms of standard deviation in 100 pings) round-trip latency between them.
We use this deployment to measure the overheads incurred by our protocol in a distributed setting, and to calibrate parameters of a simulated version of the system.

Indeed, due to the high number of configurations to evaluate, we also implement a simulated version of \sysname and of baselines:
all experiment design points presented in this paper represent around 40k protocol runs and 1TB of requests' lifecycle traces.

Protocols are simulated using Omnet++~\cite{vargaOverviewOMNeTSimulation2008}, a discrete event simulator.
Only the protocols involved during requests' lifecycle are simulated: 
in practice, TEE remote attestation can be performed as an independent first phase between each consumer-provider pair.
The search phase, which consists of many request-responses interactions, may start after this attestation.
Therefore, remote attestation is left out of simulations.
During the search phase, as in our assumptions in \Cref{sec:problem_statement}, providers receive the same $\textbf{<NEW>}$ messages at the same time.
Additionally, we do not consider the delay introduced by handling these messages, i.e., they are available instantly when they arrive at providers, to be used in subsequent requests.

The code for the distributed and simulated implementations, as well as the analysis scripts, are available publicly~\cite{coolTEEcode}.

\subsection{Experiment configurations and baselines}

Based on the implementation and networking dataset, we define experiment configurations that differ across the following dimensions.

\subsubsection{Simulation parameters}

Simulation experiments are run with a total of $N_{C}=100$ consumers and $N_{P}=8$ providers.
Providers each serve requests at a rate of 160 requests per second, i.e., they take $D=6.25ms$ to handle a request.
This maximum throughput value was determined based on performance experiments of DeSearch~\cite{liBringingDecentralizedSearch2021} in our deployment environment.

Consumers send requests at equal rates $\lambda_{r}$, such that the total request rate is $\lambda_{T}=N_{C}\lambda_{r}=\frac{\rho}{D} N_{P}$, with $\rho$ the target overall system load. 
Inter-request emission delays are exponentially distributed with parameter $\lambda_{r}$.

Assets arrive on the market at an average rate of $\lambda_{a}=100$ assets per second, according to an exponential distribution $\mathit{Exp}(\lambda_{a})$. 

We also discuss results of experiments run either with periodic asset arrivals, or periodic inter-request emission delays, but they will not be illustrated in the paper.

\subsubsection{Network topologies}
We consider two network topology types: (1)``same datacenter'' topology, where latencies are considered equal between all actors and (2) ``PlanetLab'' topology, where latencies are heterogeneous and assigned using the PlanetLab dataset~\cite{zhuNetworkLatencyEstimation2017}. This dataset represents the average roundtrip latencies between 490 nodes of the PlanetLab network, positioned across the globe. 


%\subsubsection{Network topologies}
%We run experiments under one of two network topology types: respectively with equal and heterogeneous latencies. 
%In the equal-latency setup, hereafter called ``same datacenter'' topology, all actors, whatever their role and behaviour (consumers or providers, honest or malicious), are colocated in the same datacenter: $d_{net}=0$ between all pairs of actors. \etienne{this is contradictory with section 6.2 that mentions a latency of 27 ms between nodes (pretty high for the same network btw, make sure this is not 26 us).}
%In the heterogeneous-latency setup, referred to as the ``PlanetLab'' topology, we use the PlanetLab dataset~\cite{zhuNetworkLatencyEstimation2017} to setup our network topologies. 
%This dataset represents the average roundtrip latencies between 490 nodes of the PlanetLab network, positioned across the globe. $N_{C}$ consumers and $N_{P}$ providers are each randomly and uniformly assigned a node in the topology defined by the PlanetLab dataset. Colocation of actors is permitted: multiple actors can be placed on the same node on the PlanetLab topology, in which case $d_{i,j}=0$ for colocated consumer $i$ and provider $j$. 


\subsubsection{Attack scenarios}

In the following experiments, malicious providers' proportion $p_{M}$ is in $[0,1]$, while malicious consumer's $c_{M}$ is set to 50\%, for more intuitive result analysis: 
in the fault-free case without malicious providers, neither the malicious or the honest consumers should have an advantage over the other if they are present with equal proportions. 

The protocol can be attacked according to the strategies defined in \Cref{sec:problem_statement}. 
Experiments configurations are labelled with the name of the attack strategy, or ``fault-free'' if no attack is performed.

\textit{Timing attacks} require a malicious artificial delay parameter $\delta_{att}$, which we set to $\delta_{att}=50ms$ in the experiments with \textit{timing attacks}. 
This choice was made based on calibration experiments and will be discussed in \Cref{ssec:res-malPL}.

\subsubsection{Provider selection policies}
Consumers select providers to handle an individual request using one of the following policies: 
a uniformly random selection of a provider among the $N_{P}$ providers, referred to as ``DeSearch-like'' selection, 
or a selection based on \sysname's policy, locally optimizing the selection of providers to minimize end-to-end latencies, referred to as ``COoL'' selection.

At the provider-side, in the same-datacenter setup, we also evaluate the impact of \sysname with the state-of-the-art Power-of-Two (PoT) load balancing protocol~\cite{mitzenmacherPowerTwoChoices2001}. 
In particular, we evaluate spatial Power-of-Two~\cite{panigrahyAnalysisEvaluationProximitybased2022} (sPoT), which selects the least-loaded of a consumer's two closest providers.
We also consider a variant of \sysname augmented with PoT, which we call, which we call ``COoL-PoT''.
In COoL-PoT, a consumer's \emph{Provider Selection Module} randomly selects $k=2$ distinct providers (proportionally to their selection ratios $r_{j}$).
Both providers will exchange with each other their respective queue lengths at the request's arrival: the provider with the shortest queue handles it, while the other drops the request. 
Ties are broken deterministically, e.g., based on the lowest-valued hash of the provider's ID and the request's nonce (similarly to rendezvous hashing~\cite{663936}).

Note that this new mechanism introduces a PoT specific attack, which we refer to as a ``Queue attack'' (or ``Queue-T attack'' when combined with timing attacks).
In a queue attack, malicious providers keep a secondary queue outside the TEE, so that the TEE's queue always remains at a length $L\leq1$ (considering a request being handled remains at the front of the queue until served).
This means that from the point of view of honest providers, malicious providers more often have shorter queues than them and consequently, will yield service of a request to them more often.

We evaluate the state-of-the-art solutions that rely on sending the same request to multiple providers as a class of ``multiprovider'' selection policies, with $k$ the number of providers selected for the same request content.
Only honest consumers perform this multiple provider selection, aiming to reduce IFR disadvantages.
Malicious consumers only send requests to one destination (i.e., $k=1$ in all cases), because they are treated the same way by all providers.
\section{Results}\label{sec:results}

\section{Related work}\label{sec:related_work}

The problem tackled by \sysname~is at the crossroads of multiple research domains, which we present and compare to our work in this section.

\subsection{Decentralized search mechanisms}

There are still relatively few works on decentralized search over decentralized resources, compared to those on more centralized approaches.
Moreover, their focus lies on the search itself and its properties (e.g., censorship- and bias-resistance), rather than on information front-running.
DHT-based solutions like HypeerCube~\cite{zichichi_towards_2021} and Ditto~\cite{keizerDittoDecentralisedSimilarity2023} do not guarantee result completeness but only a response from the fastest node. 
Censorship is particularly possible in HypeerCube~\cite{zichichi_towards_2021}, where only one or a few nodes are responsible for specific keywords, and can choose to not respond to requests, or to delay them.
TheGraph~\cite{Graph} is an indexing mechanism for Ethereum smart contracts, where resources are indexed based on their perceived value by curators. 
Again, completeness is not guaranteed.
Finally, DeSearch~\cite{liBringingDecentralizedSearch2021} provides completeness and integrity of results using TEEs, but is not resilient as-is to information front-running, as search providers have the power over the network interfaces that connect the TEE to the outside world.

\subsection{Transaction-front-running-resilient protocols}

\etienne{this section could be reduced significantly (the RW in general is a tad long).}
Some mechanisms deployed by protocols that protect against transaction front-running (TFR) in blockchain could be used against information front-running (IFR) in search systems, as both concepts are related: 
while the former aims to order a specific transaction before others, the latter delays some responses in favor of others.
Note that while solutions against TFR bear similarities with IFR in that they also play on the ordering and timing of events, IFR remains a different problem from TFR: TFR solutions cannot be used as-is to protect against IFR. 
We discuss below the extent to which their mechanisms can be repurposed to protect against IFR.

Torres et al.~\cite{torresFrontrunnerJonesRaiders2021} present the absence of transaction confidentiality and the blockchain miner's ability to choose transaction order as key causes for TFR.
To this end, transaction-ordering systems were proposed~\cite{bentovTesseractRealTimeCryptocurrency2019,kelkarThemisFastStrong2023,stathakopoulouAddingFairnessOrder2021}, ensuring that individual servers cannot influence the ordering of transactions.
Translating their protocols to our context, given such ordering mechanisms, existing acknowledgment messages sent back to the client can be repurposed as a response message to the client's request.
The response time after computation will then be that of the fastest server in the network. 
If the closest servers are malicious, then the client can still suffer from timing attacks, as is the case with \sysname.
Nonetheless, requiring consensus for a request-response scheme is computationally expensive (considering $n$ nodes, there is at least $O(n^2)$ in message complexity with TOB~\cite{cachinQuickOrderFairness2022}; or $O(n)$ optimistically with SNARK-THEMIS~\cite{kelkarThemisFastStrong2023}, but with added SNARK-proof computation time).

Fairy~\cite{stathakopoulouAddingFairnessOrder2021} also uses transaction sender anonymization through relays like TOR~\cite{TorProjectPrivacy}. 
Because malicious consumers can still decide to communicate directly with providers, they can still be recognized by malicious providers.
Meanwhile, honest responses are delayed through the high-latency layered network. 
As \Cref{ssec:res-malPL} shows, this creates an IFR disavantage for honest consumers by design, even without attacks.

The behavior of solutions presented above, which are originally tailored to protect against \emph{transaction front-running}, is captured by the ``multiprovider selection'' class defined in \Cref{sec:experimental_protocol}.
We have shown in \Cref{ssec:res-multi} that IFR has a strong impact in multiprovider-selection-based solutions.

\subsection{Oracle systems}

Blockchains operate as closed systems: they must rely on other entities to access data that is located outside their storage structure. 
``Oracle systems'' provide this dataflow between blockchains and the outside world.
Request-response search systems on top of blockchains~\cite{liBringingDecentralizedSearch2021,keizerDittoDecentralisedSimilarity2023,zichichi_towards_2021} can be seen as ``pull-outbound oracles''~\cite{muhlbergerFoundationalOraclePatterns2020}: they draw data from the blockchain and feed it to outside entities, in our case, the market consumers.
%Meanwhile, publish-subscribe search systems, left for future work, would be similar to push-outbound oracles.

If we consider the potential usage of our contribution for inbound oracles, i.e., those that feed data to blockchains, the block generation period (e.g., around 12s in Ethereum 2.0~\cite{etherscan.ioEthereumAverageBlock}) outweighs the subsecond gains from latency optimization.
Additionally, there is no obvious translation of inter-consumer competition in that context, so no IFR and therefore, no need to protect against it. 
This is why our proposed solution \sysname is rather suited for outbound oracles, that is, to serve users or systems that are not blockchains.
For more details, recent surveys explore existing works on blockchain oracle systems~\cite{caldarelliOverviewBlockchainOracle2022,al-breikiTrustworthyBlockchainOracles2020,heissOraclesTrustworthyData2019}. 
%The latter focuses on categorizing academic contributions to the field, while the first two propose an oracle system taxonomy and assess both industry and academic works. 

\subsection{Decentralized QoS-based provider selection}

Provider selection protocols are more in line with our work, as they indirectly tackle the IFR problem.
However, they usually operate in a non-adversarial setting, where providers are assumed to be honest and the only problem is to select the best one. 
In most cases, \sysname\space uses similar mechanisms to select providers, but adapts them to the adversarial setting with a trusted actor which is the TEE.
A close work with ours is Go-with-the-winner~\cite{liuGowiththewinnerPerformanceBased2016}, where providers are selected based on their end-to-end latency, in a non-adversarial setting, without insights from the providers themselves. 
%This protects the protocol from being influenced by malicious providers. 
We augment this work to a setting with malicious actors: we leverage TEEs notably to protect against content attacks, and use COoL selection against timing attacks.
sPOT~\cite{panigrahyAnalysisEvaluationProximitybased2022}, Spatial Power of Two, is based on a well-known mechanism called Power of Two Choices~\cite{mitzenmacherPowerTwoChoices2001}, where a client selects two providers at random and chooses the least loaded one.
In the case of sPOT, the two providers are the closest ones to the client. 
We show in \Cref{ssec:res-spot} that this protocol is subject to IFR attacks.
In \sysname, we use historic latencies instead of network distances to select providers.
%With \sysname, the TEE is able to provide a trustworthy indicator of its historic load, to use for later selection.
Other works, like DONAR~\cite{wendellDONARDecentralizedServer2010}, use an additional role in the protocol, a mapper or broker, to select providers. 
This adds communication costs that can be bypassed by malicious consumers, increasing by design their IFR advantage.
%\subsection{Tackling fair Quality-of-Service}
%What they provide as additional properties and at what cost (if any)?
\section{Conclusion}\label{sec:conclusion}

In this paper, we highlight how providers of a decentralized marketplace's search service are able to give favorable treatment to part of the userbase, through \emph{information front-running} attacks, even if Trusted Execution Environments (TEEs) are used to guarantee the integrity and confidentiality of the search mechanism.

We show that quality-of-service-aware provider selection as performed by our proposed mechanism \sysname, on top of TEEs, is paramount to counter such malicious provider behaviour. 
Thanks to \sysname, we show that whether users collude or not with malicious providers, they do not get an edge over others in terms of fresher market state information, as long as there is enough honest computing power to serve all requests, in networks with low latency heterogeneity.
Future research directions include investigating the dynamic spawning or retirement of service providers over the network topology, tailored to the users' latency requirements to deal with the intrinsic heterogeneity of client latency distributions.

%%
%% The next two lines define the bibliography style to be used, and
%% the bibliography file.
\bibliographystyle{ACM-Reference-Format}
\bibliography{imports/bibliography}



\end{document}
\endinput
%%
%% End of file `sample-sigplan.tex'.
